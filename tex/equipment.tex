\chapter{装置} \label{equipment}
\section{探索装置の概要}
今回の実験に用いた探索装置の概略図及び座標軸を図4.1に、全体図を図4.2に示す。
実験装置は、2cmのアルミ板を2cm間隔で8段積み上げている。アルミ板の各層には、通過粒子のxy平面での座標を特定するためプラスチックシンチレータを7度傾けたものを2層ずつ配置している。
この2次元飛跡検出器をz方向に8段重ねることで3次元飛跡検出器とし反応の探索を行った。
プラスチックシンチレータからの光は波長変換ファイバーによって伝えられ、光検出器であるMPPCを用い検出しEASIROC MODULEを用いてデジタル信号として読み出した。
また検出装置に入射する$\mu$粒子を制限するため、装置上部にプラスチックシンチレータを3枚配置しトリガーシンチレータとして用いた。

\section{装置詳細}
\subsection{プラスチックシンチレータ}
反応の探索に用いたプラスチックシンチレータの大きさは厚さ1cm、長さ75cm、幅4cmである。
プラスチックシンチレータは荷電粒子が通過すると電子が励起し、基底状態に戻るときにシンチレーション光と呼ばれる光を発する荷電粒子検出器である。
このプラスチックシンチレータを4枚を2層ずつ8段配置し、計64枚使用している。

\subsection{波長変換ファイバー}
波長変換ファイバーはプラスチックシンチレータからの光を吸収し、光検出器の感度がいい波長(400~500nm)に変換する。
直径1.2mmのものを使用し、装置両側の基盤に取り付けた光検出器に光を導いた。

\subsection{MPPC}
光検出器として、MPPC(Multi Pixel Photon Counter)を用いた。
MPPCは受光面が複数のピクセルからなるフォトンカウンティングデバイスであり、高い増幅率を持つ半導体光検出器である。
各ピクセルにおいて降伏電圧以上で光電子をアバランシェ増幅し、ピクセル内に発生した光電子数によらず同じ波高を出力する。
さらに複数のピクセルで発生した波高は重ね合わせて出力される。
この波高から検出した光電子数を見積もることができる。
このMPPC64個を基板に取り付け、ゴミコネクタを用いてプラスチックシンチレータの片側から出ている波長変換ファイバーと接続した。

\subsection{EASIROC}

\subsection{トリガーシンチレータ}
\subsection{アルミ板}

\section{データ取得のセットアップ}

\section{MPPCにおけるゲインの個体差}