\chapter{宇宙線ミューオンの解析}\label{analysis}
測定データを用いてどのように光生成反応イベントを探索したかを説明する.

\section{解析に用いたデータ}\label{sec:anal:data}
表\ref{tab:analyzed_data}に示すようにトリガーレート15\ cps, データ取得時間約260時間, 取得イベント数12054470イベントのデータを解析に用いた.
\begin{table}[H]
    \centering
    \caption{解析に用いたデータ}
    \label{tab:analyzed_data}
    \begin{tabular}{|c|r|}
        \hline
        トリガーレート & 15 cps    \\ \hline
        取得時間       & 約260時間 \\ \hline
        イベント数     & 12054470  \\ \hline
    \end{tabular}
\end{table}

\section{しきい値の設定}\label{sec:anal:threshold}
\ref{sec:mppc_gain_diff}節の通り, ADC値は荷電粒子がシンチレータに渡したエネルギーに比例している量であるのでADC値に適切なしきい値を設けることでシンチレータ毎に荷電粒子が通過したかどうかを判別できる.
図\ref{fig:mu_mppc}に示す分布のペデスタルによるピークからある程度離れた値をしきい値とすることにした.
具体的には図\ref{fig:adc_eff}のようにADC値のしきい値を800から1500まで1刻みで変更した時の検出効率を計算し, チャンネル毎にeffeciencyが大きく下がらない程度のしきい値を探すことでしきい値を決定した.
図\ref{fig:adc_eff}, \ref{fig:threhist} の緑線は実験に用いたADC値のしきい値を示している.
\begin{figure}[H]
    \centering
    \includegraphics[height=5.0cm]{img/adc_eff.png}
    \caption{しきい値と検出効率の関係}
    \label{fig:adc_eff}
\end{figure}
\begin{figure}[H]
    \centering
    \includegraphics[height=5.0cm]{img/pedestal.png}
    \caption{宇宙線による ADC値の分布}
    \label{fig:threhist}
\end{figure}

\section{ヒット情報の作成}
\subsection{ヒット情報の作成方法}\label{subsec:anal:make_hit}
\ref{sec:anal:threshold} 節で荷電粒子が64枚あるシンチレータのうちのどのシンチレータを通過したかという情報を得ることができた.図\ref{fig:igata01}は実験装置のある層でシンチレータに荷電粒子が走ったか/走っていなかったかを色分けしたものである.図の赤く塗られている部分のように, 荷電粒子はしきい値を超えたシンチレータ同士が重なる位置を通過したと考えることができる.図\ref{fig:igata01}は1層分の絵であるが, 8層分に以上の処理を施すことでで荷電粒子の3次元飛跡のようなものの情報が得られた(図\ref{fig:eventdisplay}).図\ref{fig:eventdisplay}の透過している青い領域は検出器の有感領域を示しており, 黒い領域は解析によって荷電粒子がと通過したと判定された領域である.この3次元飛跡のようなものから飛跡を再構成することができる.
\begin{figure}[H]
    \begin{tabular}{cc}
        \begin{minipage}[t]{0.45\hsize}
            \centering
            \includegraphics[height=5.0cm]{img/igata_01.png}
            \caption{1層分のシンチレータ}
            \label{fig:igata01}
        \end{minipage}
        \begin{minipage}[t]{0.45\hsize}
            \centering
            \includegraphics[height=5.0cm]{img/igata_02.png}
            \caption{1層に2つ以上の荷電粒子が通ったとき}
            \label{fig:igata02}
        \end{minipage}
    \end{tabular}
\end{figure}
\begin{figure}[H]
    \centering
    \includegraphics[height=5.0cm]{img/eventdisplay.png}
    \caption{イベントディスプレイ}
    \label{fig:eventdisplay}
\end{figure}
\subsection{ヒット情報作成の問題点}
\ref{subsec:anal:make_hit}節で説明したヒット情報の計算方法には問題点が存在する.
1層に2個以上の荷電粒子が通過した場合に作成されたヒット情報からは飛跡を再構成できないという点である.
例えば, 図\ref{fig:igata02}に示すようにシンチレータから得られたADC値がしきい値を超えた場合は赤く塗られる箇所(荷電粒子が通過したと考えられる場所)が余分にできてしまい, 荷電粒子の本当の飛跡がわからなくなってしまう.本実験では光生成反応を探索しているため探索対象は複数の粒子の飛跡であるが, この問題点は複数の粒子の飛跡の再構成を難しくしてしまっている.

\section{光生成反応イベントの探索}
\subsection{イベントセレクション}\label{sec:anal:eventcut}
\ref{sec:anal:data}に示した測定データから光生成反応と考えられるイベントを以下に示す条件によって選んだ.
\begin{itemize}
    \item[(1)] 6層以上の層を荷電粒子が通った
    \item[(2)] 1, 2, 8層目で1つの荷電粒子が通った
    \item[(3)] 3〜7層目で荷電粒子の反応が2つ以上ある
\end{itemize}
(1), (2)の条件は荷電粒子が検出器内をある程度真っ直ぐ通ったことを保証するための条件である.(3)の条件は光生成反応によって期待される複数粒子の飛跡を捉えるための条件である.

イベントセレクションの結果残ったイベントは2543イベントであり, 総取得イベント数12054473の約0.021\%であった.イベントセレクションによって得られたイベントの例を図\ref{fig:eventcuted_event}に示す.図\ref{fig:eventcuted_event}は上から順にz軸正の方向から各層を示していて, 一つの菱形がピクセルを表しており, 黒く塗りつぶされたピクセルが荷電粒子が検出器を通ったと考えられるピクセルである.図\ref{fig:eventcuted_event}を見ると, ミューオンがz軸正の方向から検出器に入射し, 3層目あたりでもう一つの荷電粒子が発生したと見ることができる.しかし, このイベントが光生成反応によって得られたイベントであるかどうかは定かではない.
\begin{figure}[H]
    \centering
    \includegraphics[height=5.0cm]{img/eventcutted_figure.png}
    \caption{イベントセレクション後のイベントの例}
    \label{fig:eventcuted_event}
\end{figure}

\subsection{イベントセレクションの検証}\label{sec:eventcut_val}
\ref{sec:anal:eventcut}節のイベントセレクションによって得られたイベントは光生成反応由来かどうか定かではないと述べた.そこで, シミュレーションによって得られたデータを\ref{sec:anal:eventcut}節と同じイベントセレクションで解析することによってイベントセレクションによってどれだけの光生成反応イベントが得られたかどうかを検証した.結果を表\ref{tab:eventcut_val}に示す.
\begin{table}[H]
    \centering
    \caption{イベントセレクションの検証}
    \label{tab:eventcut_val}
    \begin{tabular}{|c|c|}
        \hline
        \begin{tabular}[c]{@{}c@{}}全イベント\\ 10000000\end{tabular}                                        & \begin{tabular}[c]{@{}c@{}}カット後\\ 5759\end{tabular}                                        \\ \hline
        \begin{tabular}[c]{@{}c@{}}全イベント中でパイオンを含むイベント\\ 19 (純粋度 0.00019\%)\end{tabular} & \begin{tabular}[c]{@{}c@{}}カット後でパイオンを含むイベント\\ 14  (純粋度 0.24\%)\end{tabular} \\ \hline
    \end{tabular}
\end{table}
表\ref{tab:eventcut_val}における純粋度はイベントセレクションによって得られたイベントのうちパイオンを含むイベントの割合である.
表\ref{tab:eventcut_val}によるとイベントセレクション前とイベントセレクション後でパイオンイベントの純粋度は0.00019\%から0.24\%に上昇していることがわかる.しかし, カット後のイベントにおけるパイオンイベントの純粋度が0.24\%というのはイベントセレクションによって得られたイベントの99\% 以上にパイオンでないイベント (バックグラウンド) を含んでしまっていることを意味している.よって, \ref{sec:anal:eventcut}節で行ったイベントセレクションではバックグラインドが十分に取り除くことができなかったといえる.

\subsection{バックグラウンドの検討}\label{sec:anal:background}
図\ref{fig:anal:bg01}は\ref{sec:eventcut_val}節で使用したシミュレーションデータから, イベントセレクションがパイオンイベントと間違えたイベントを可視化したものの例である.図\ref{fig:anal:bg01}に示すようなミューオンが検出器内の電子を弾くイベントが支配的であった.図\ref{fig:pion_electron_degrere}はシミュレーションによって得られた, ミューオンによって弾き出された電子とパイオンの角度分布である.青色, 赤色で表されるヒストグラムはそれぞれ光生成反応によって生成されたパイオン, ミューオンによって蹴り出された電子の散乱角分布を示している.角度は上空方向を正としたベクトル(図\ref{fig:arrangement}におけるz軸)と粒子の運動量ベクトルの内積によって定義される.本研究において用いたイベントセレクションでは主に, 入射ミューオンと同じ方向へ飛ぶパイオンを検出しようとしたために図\ref{fig:pion_electron_degrere}における$\pi$ラジアン付近の探索をしていた.そのために電子散乱におけるイベントがイベントセレクション後のイベントに含まれていたと考えられる.
\begin{figure}[H]
    \centering
    \includegraphics[height=5.0cm]{img/anal_bg01.png}
    \caption{バックグラウンドと考えられるイベント}
    \label{fig:anal:bg01}
\end{figure}
\begin{figure}[H]
    \centering
    \includegraphics[height=5.0cm]{img/pi_e_degree.png}
    \caption{パイオンと電子の角度分布}
    \label{fig:pion_electron_degrere}
\end{figure}