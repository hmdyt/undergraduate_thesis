\chapter{解析}\label{analysis}
測定データを用いてどのように荷電粒子の飛跡の推定を行ったかを説明する.

\section{解析に用いたデータ}\label{sec:anal:data}
表\ref{tab:analyzed_data}に示すようにトリガーレート15\ cps, データ取得時間約260時間, 取得イベント数12054470イベントのデータを解析に用いた.
\begin{table}[H]
    \centering
    \caption{解析に用いたデータ}
    \label{tab:analyzed_data}
    \begin{tabular}{|c|r|}
        \hline
        トリガーレート & 15 cps    \\ \hline
        取得時間       & 約260時間 \\ \hline
        イベント数     & 12054470  \\ \hline
    \end{tabular}
\end{table}

\section{しきい値の設定}\label{sec:anal:threshold}
測定によって得られたデータは各チャンネル毎に12bit (4096段階) の強度を持っている.ここでは12bitを0から4095までの整数で表すことにし, その値のことを ADC value と呼ぶ.
ADC value はMPPCから得られた信号をEASIROCが増幅, 整形しAD変換を行うことで得られた値である.
ADC value はMPPCからの信号強度, すなわち荷電粒子がシンチレータに渡したエネルギーに比例している量であるのでADC valueに適切なしきい値を設けることでシンチレータ毎に荷電粒子が通過したかどうかを判別できる.
シンチレータがエネルギーを受け取らなかった時のADC value は本実験で用いたEASIROCにおいては典型的に800付近であったため, ADC value 800付近からある程度離れた値をしきい値とすることにした.
具体的には図\ref{fig:adc_eff}のようにADC value のしきい値を800から1500まで1刻みで変更した時の検出効率を計算し, チャンネル毎にeffeciencyが大きく下がらない程度のしきい値を探すことでしきい値を決定した.
図\ref{fig:adc_eff}, \ref{fig:threhist} の赤線は実験に用いたADC valueのしきい値を示している。
\begin{figure}[H]
    \centering
    \includegraphics[height=5.0cm]{img/adc_eff.png}
    \caption{しきい値と検出効率の関係}
    \label{fig:adc_eff}
\end{figure}
\begin{figure}[H]
    \centering
    \includegraphics[height=5.0cm]{img/pedestal.png}
    \caption{宇宙線による ADC value の分布}
    \label{fig:threhist}
\end{figure}

\section{ヒット情報の作成}
\subsection{ヒット情報の作成方法}\label{subsec:anal:make_hit}
\ref{sec:anal:threshold} 節で荷電粒子が64枚あるシンチレータのうちのどのシンチレータを通過したかという情報を得ることができた.図\ref{fig:igata01}は実験装置のある層でシンチレータに荷電粒子が走ったか/走っていなかったかを色分けしたものである.図の赤く塗られている部分のように, 荷電粒子はしきい値を超えたシンチレータ同士が重なる位置を通過したと考えることができる.図\ref{fig:igata01}は1層分の絵であるが, 8層分に以上の処理を施すことでで荷電粒子の3次元飛跡のようなも情報を得られた(TODO: 3次元の絵).この3次元飛跡のようなものから飛跡を再構成することができる.
\begin{figure}[H]
    \centering
    \includegraphics[height=5.0cm]{img/igata_01.png}
    \caption{1層分のシンチレータ}
    \label{fig:igata01}
\end{figure}
\subsection{ヒット情報作成の問題点}
\ref{subsec:anal:make_hit}節で説明したヒット情報の計算方法には問題点が存在する.
1層に2個以上の荷電粒子が通過した場合に作成されたヒット情報からは飛跡を再構成できないという点である.
例えば, 図\ref{fig:igata02}に示すようにシンチレータから得られたADC valueがしきい値を超えた場合は赤く塗られる箇所(荷電粒子が通過したと考えられる場所)が余分にできてしまい, 荷電粒子の本当の飛跡がわからなくなってしまう.本実験では光生成反応を探索しているため探索対象は複数の粒子の飛跡であるが, この問題点は複数の粒子の飛跡の再構成を難しくしてしまっている.
\begin{figure}
    \centering
    \includegraphics[height=5.0cm]{img/igata_02.png}
    \caption{1層に2つ以上の荷電粒子が通ったとき}
    \label{fig:igata02}
\end{figure}

\section{光生成反応イベントの探索}
\subsection{イベントセレクション}\label{sec:anal:eventcut}
\ref{sec:anal:data}に示した測定データから光生成反応と考えられるイベントを以下に示す条件によってカットをかけた.
\begin{itemize}
    \item[(1)] 6層以上の層を荷電粒子が通った
    \item[(2)] 1, 2, 8層目で1つの荷電粒子が通った
    \item[(3)] 3〜7層目で荷電粒子の反応が2つ以上ある
\end{itemize}
(1), (2)の条件は荷電粒子が検出器内をある程度真っ直ぐ通ったことを保証するための条件である.(3)の条件は光生成反応によって期待される複数粒子の飛跡を捉えるための条件である.

イベントセレクションの結果残ったイベントは2543イベントであり, 総取得イベント数12054473の約0.021\%であった.イベントセレクションによって得られたイベントの例を図\ref{fig:eventcuted_event}に示す.図\ref{fig:eventcuted_event}は上から順にz軸正の方向から各層を示していて, 一つの菱形がピクセルを表しており, 黒く塗りつぶされたピクセルが荷電粒子が検出器を通ったと考えられるピクセルである.図\ref{fig:eventcuted_event}を見ると, ミューオンがz軸正の方向から検出器に入射し, 3層目あたりでもう一つの荷電粒子が発生したと見ることができる.しかし, このイベントが光生成反応によって得られたイベントであるかどうかは定かではない.
\begin{figure}[H]
    \centering
    \includegraphics[height=5.0cm]{img/eventcutted_figure.png}
    \caption{イベントセレクション後のイベントの例}
    \label{fig:eventcuted_event}
\end{figure}

\subsection{イベントセレクションの検証}\label{sec:eventcut_val}
\ref{sec:anal:eventcut}節のイベントセレクションによって得られたイベントは光生成反応由来かどうか定かではないと述べた.そこで, シミュレーションによって得られたデータを\ref{sec:anal:eventcut}節と同じイベントセレクションで解析することによってイベントセレクションによってどれだけの光生成反応イベントが得られたかどうかを検証した.結果を表\ref{tab:eventcut_val}に示す.
\begin{table}[H]
    \centering
    \caption{イベントセレクションの検証}
    \label{tab:eventcut_val}
    \begin{tabular}{|c|c|}
        \hline
        \begin{tabular}[c]{@{}c@{}}全イベント\\ 10000000\end{tabular}                      & \begin{tabular}[c]{@{}c@{}}カット後\\ 5759\end{tabular}                       \\ \hline
        \begin{tabular}[c]{@{}c@{}}全イベント中でパイオンを含むイベント\\ 313\end{tabular} & \begin{tabular}[c]{@{}c@{}}カット後でパイオンを含むイベント\\ 14\end{tabular} \\ \hline
    \end{tabular}
\end{table}
表\ref{tab:eventcut_val}によるとイベントセレクション前とイベントセレクション前でパイオンイベントの純粋度は0.0031\%から0.24\%に上昇していることがわかる.しかし, カット後のイベントにおけるパイオンイベントの純粋度が0.24\%というのはイベントセレクションでは取り除けなかったバックグラウンドの存在が示唆される.よって, \ref{sec:anal:eventcut}節で行ったイベントセレクションではバックグラインドが十分に取り除くことができなかったといえる.

\subsection{バックグラウンドの検討}\label{sec:anal:background}
図\ref{fig:anal:bg01}は\ref{sec:eventcut_val}節で使用したシミュレーションデータから, イベントセレクションがパイオンイベントと間違えたイベントを可視化したものの例である.図\ref{fig:anal:bg01}に示すようなミューオンが検出器内の電子を弾くイベントが支配的であった.このような電子散乱では電子の質量がミューオンに比べ軽いため電子は前方に散乱する.対して, 光生成反応によるパイオンイベントではパイオンは後方に散乱することがある.このような違いをうまく用いることで, パイオンイベントの純粋度を上げるようなイベントセレクションを実現することができると考えられる.
\begin{figure}[H]
    \centering
    \includegraphics[height=5.0cm]{img/anal_bg01.png}
    \caption{バックグラウンドと考えられるイベント}
    \label{fig:anal:bg01}
\end{figure}