\chapter{装置} \label{equipment}
\section{探索装置の概要}
今回の実験に用いた探索装置の概略図及び座標軸を図\ref{fig:arrangement}に, $Y$方向から見た図を図\ref{fig:y_arrangement}に示す.


\begin{figure}[H]
    \centering
    \includegraphics[height=5cm]{img/equipment.jpg}
    \caption{探索装置の概略図}
    \label{fig:arrangement}
\end{figure}

\begin{figure}[H]
    \centering
    \includegraphics[height=6cm]{img/equipment_y.jpg}
    \caption{Y方向から見た装置}
    \label{fig:y_arrangement}
\end{figure}

実験装置は, 2cmのアルミ板を2cm間隔で8段積み上げている.
アルミ板の各層には, 通過粒子の$xy$平面での座標を特定するため、図\ref{fig:fiber}のようにプラスチックシンチレータを逆側に7度ずつ傾けたものを2層ずつアルミの間に配置している.
この2次元飛跡検出器を$z$方向に8段重ねることで3次元飛跡検出器とし反応の探索を行った.
プラスチックシンチレータからの光は波長変換ファイバーによって伝えられ, 光検出器であるMPPCを用い検出しEASIROC MODULEを用いてデジタル信号として読み出された.
また検出装置に入射する$\mu$粒子を検出装置の上方から入射し検出器を通過するものに制限するため, 装置上部にプラスチックシンチレータを3枚配置しトリガーシンチレータとして用いた.

\begin{figure}[H]
    \centering
    \includegraphics[height=7cm]{img/Scinti.jpeg}
    \caption{プラスチックシンチレータと光ファイバー}
    \label{fig:fiber}
\end{figure}

\section{装置詳細}
\subsection{プラスチックシンチレータ}
反応の探索に用いたプラスチックシンチレータの大きさは厚さ1cm, 長さ75cm, 幅4cmである.
プラスチックシンチレータは荷電粒子が通過すると電子が励起し, 基底状態に戻るときにシンチレーション光と呼ばれる光を発する荷電粒子検出器である.
このプラスチックシンチレータを4枚を2層ずつ8段配置し, 計64枚使用している.

\subsection{波長変換ファイバー}
波長変換ファイバーはプラスチックシンチレータからの光を吸収し, 光検出器の感度が良い波長(400~500nm)に変換する.
直径1.2mmのものを使用し, 装置片側の基盤に取り付けた光検出器にプラスチックシンチレータからの光を導いた.
図\ref{fig:fiber}のように2枚重ねたシンチレータは逆側から光を導いた.

\subsection{MPPC}
光検出器として, MPPC(Multi Pixel Photon Counter)を用いた.
MPPCは受光面が複数のピクセルからなるフォトンカウンティングデバイスである.各ピクセルは高い増幅率を持つ半導体光検出器である.
各ピクセルにおいて降伏電圧以上で光電子をアバランシェ増幅し, ピクセル内に発生した光電子数によらず同じ波高を出力する.
ピクセルは複数あり, 波高は重ね合わせて出力される.
この波高から検出した光電子数を見積もることができる.
このMPPC64個を基板に取り付け, ゴミコネクタを用いてプラスチックシンチレータの片側から出ている波長変換ファイバーと接続した.
ゴミコネクタはプラスチックの2つのパーツからなるコネクタでファイバーの径と同じ1.5mmの穴が開いており、ファイバーとMPPCを密着して接続することで光漏れを防ぐ。

\begin{figure}[H]
    \begin{minipage}[b]{0.47\linewidth}
        \centering
        \includegraphics[height=4.5cm]{img/1_mppc.jpg}
        \caption{MPPC}
        \label{fig:mppc}
    \end{minipage}
    \begin{minipage}[b]{0.47\linewidth}
        \centering
        \includegraphics[height=4.5cm]{img/avalanche.jpg}
        \caption{アバランシェ増幅}
        \label{fig:avalanche}
    \end{minipage}
\end{figure}

\subsection{EASIROC}
MPPCからの信号の読み出しにEASIROC(Extended Analogue Silicon PM Integrated Read Out Chip)を用いた.
最大64chのMPPCへの電圧の印加や同時読み出しが可能である.

\begin{figure}[H]
    \centering
    \includegraphics[height=6cm]{img/mppc_easiroc.jpg}
    \caption{EASIROCを用いて読み出したMPPCの信号}
    \label{fig:easiroc}
\end{figure}

Amp, Shaper, Discriminator が内蔵されており, 光電子数の情報をADC(Analog to Digital Converter)で, 時間情報を TDC(Time to Digital Converter) で取得できる.
\\
図\ref{fig:easiroc}はMPPCにLEDの光をあて, EASIROCを用いて読み出した信号の例である.
\\
MPPC にかける電圧はEASIROCによる全チャンネル一律の値の$V_0$ と, DAC からの出力電圧との間の電圧を MPPC にかけることによりコントロールした.
DACの入力を$n_{DAC}$とすると, Input DACによりコントロールする電圧($V_i$)は,
\begin{equation}
    V_i = -0.0195(n_{DAC}) + 9.4479
\end{equation}
である.
また, $V_i$と$V_0$とBias Voltage($V$)の関係は
\begin{equation}
    V = V_0 - V_i
\end{equation}
である.
InputDACを調整することで各MPPCへのBias Voltageを調整した.

\subsection{トリガーシンチレータ}
図$\ref{fig:trigger}$は装置を上から見た写真である.
装置のすぐ上に, トリガーシンチレータとして厚さ1cm, 長さ126cm, 幅7cmのプラスチックシンチレータを3枚設置した.
トリガーシンチレータの大きさは検出器に入射する$\mu$粒子の方向を限定するため, 検出器の$x$方向の大きさより少し小さく配置している.
\begin{figure}[H]
    \centering
    \includegraphics[height=6cm]{img/Trigger.jpeg}
    \caption{トリガーシンチレータの配置}
    \label{fig:trigger}
\end{figure}

\subsection{アルミ板}
厚さ2cm, 長さ100cm, 幅30cmのアルミの板を2cm間隔で8段積み上げた.
プラスチックシンチレータの間にアルミの板を用いたのは, 装置の質量を稼ぎ反応を起こしやすくするためである.

\section{データ取得のセットアップ}
図\ref{fig:setup}は今回の実験におけるデータ取得のセットアップである.
\begin{figure}[H]
    \centering
    \includegraphics[height=6cm]{img/setup.jpg}
    \caption{データ取得のセットアップ}
    \label{fig:setup}
\end{figure}
装置上部のトリガーシンチレータのいずれかに$\mu$粒子が入射したとき, データ取得が行われる.
トリガー信号はBRoaDⅢに送られ, 測定に必要なPeak Hold, T Stop, Accept信号が適切な時間遅延させられてEASIROCに送られている.
Peak Holdに信号が入ると波高と時間の両方の測定が開始され, T Stopに信号が入ると測定が止まる.
Acceptに信号が入ると信号をデジタル値として読み出す.
読み出しの間は次のトリガーを入れてはいけないので, BroadIIIモジュールを使ってAcceptのあと$5 \mu s$トリガーを抑制している.

\section{MPPCにおけるゲインの個体差}\label{sec:mppc_gain_diff}
\begin{figure}[H]
    \centering
    \includegraphics[height=6cm]{img/mppc_gain.jpg}
    \caption{宇宙線$\mu$粒子のMPPCにおける信号}
    \label{fig:mu_mppc}
\end{figure}

図\ref{fig:mu_mppc}は宇宙線$\mu$粒子が検出器に入射したときのMPPCの信号をEASIROCを用いて読み出した時のエネルギーのヒストグラムである.
測定によって得られたデータは各チャンネル毎に12bit (4096段階) の強度を持っている.ここでは12bitを0から4095までの整数で表すことにし, その値のことをADC値と呼ぶ.
ADC値はMPPCから得られた信号をEASIROCが増幅, 整形しAD変換を行うことで得られた値である.
荷電粒子が通過したシンチレータは荷電粒子からエネルギーを受け取り, 受け取ったエネルギー分のX線を放出する.シンチレータが放出したX線は波長変換ファイバーによってMPPCへと到達する.
ADC値はオフセットを除けばMPPCからの信号強度(光電子数), すなわち荷電粒子がシンチレータに渡したエネルギーにおおむね比例している量である.
\\
EASIROCのADC値は強度0のとき850付近を示すので, 1番左のピークはペデスタルであると分かる.
ADC値3500付近は光電子数が多くサチュレーションしている.\\
MPPCに同じ電圧をかけて$\mu$粒子を測定するとシンチレータの中で$\mu$粒子が最小電離損失をして, 飛跡の長さが一定であれば, 平均値はほぼ一定になるので各チャンネルで同じところにピークが立つことが期待される.
しかしMPPCの個体差により波高が異なるので図\ref{fig:mu_mppc}における電離損失に対応するピーク(ADC値1600付近)の位置も異なる.
今回の実験では粒子が通ったと判定するためのしきい値を同じくらいの値で決定するために, 宇宙線$\mu$粒子を用いてキャリブレーションを行い信号のピークの位置を揃えた.
電圧を変えて宇宙線$\mu$粒子の信号を3回測定し, ヒストグラムをランダウ分布でフィットしそれぞれのチャンネルのピークの位置を求めて同じ電圧をかけた時にピークの位置が同じ位置になるようにInputDACの値を調整した.
図\ref{fig:gain_before}が揃える前のチャンネルごとのピークの位置で, 図\ref{fig:gain_after}が揃えた後の図である.
\begin{figure}[H]
    \begin{minipage}[b]{0.47\linewidth}
        \centering
        \includegraphics[height=5cm]{img/gain_before.jpg}
        \caption{ゲインを揃える前のピークのADC値}
        \label{fig:gain_before}
    \end{minipage}
    \begin{minipage}[b]{0.47\linewidth}
        \centering
        \includegraphics[height=5cm]{img/gain_after.jpg}
        \caption{ゲインを揃えた後のピークのADC値}
        \label{fig:gain_after}
    \end{minipage}
\end{figure}